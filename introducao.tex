\chapter{INTRODUÇÃO}
\label{intro}

Os eventos de difração trazem informações sobre as pequenas heterogeneidades em subsuperfície. Como falhas, fraturas
canais e flancos de domo de sal \cite{sep_dif}.
Portanto, apesar do processamento sísmico concentrar seus esforços no imageamento de ondas refletidas, o valor das
difrações não deve ser subestimado \cite{khaidukov}.

Em teoria, podemos separar difrações e reflexões no domínio dos dados: Ondas sísmicas refletidas e difratadas são
dois fenômenos físicos fundamentalmente distintos \cite{klen}. Partimos da hipótese de que, no volume dos dados
empilhados, correspondem a eventos fortemente coerentes com inclinações contínuas. Remover as reflexões, revelamos
as difrações.


Partindo da seção empilhada, nós propomos a identificação e remoção dos eventos de reflexão 
através do filtro de destruição de ondas planas 
\cite{claerbout}: O método de destruição de ondas planas estima a continuidade das inclinações 
locais dos eventos sísmicos dominantes
formando uma predição de cada dado no traço em relação aos traços na vizinhança com a otimização a partir de
filtros não estacionários compactos que seguem que seguem a energia sísmica ao longo das inclinações estimadas
\cite{sep_dif}.

Após a remoção das reflexões da seção empilhada, migramos a seção,
contendo apenas eventos de difração, múltiplas vezes. Para isto utilizamos
diferentes velocidades de migração em um processo chamado de \textit{velocity continuation}, a partir de um algoritmo baseado
na FFT \cite{sep_dif}.

Montamos os \textit{focusing image gathers} (FIG) calculando para cada ponto no domínio 
da imagem a sua \textit{variação máxima local},
para cada uma das imagens migradas produzidas no passo anterior a partir de uma velocidade de migração distinta. 
Para pontos no domínio da imagem que sejam pontos de foco (difrações), o valor da \textit{variação máxima local} é máximo. 
Assim o FIG é um painel que mostra os maiores valores de \textit{variação máxima local} 
para as velocidade de migração otimizadas (análise de velocidade de migração).

Obtemos a curva de tempo de trânsito do método do elemento de reflexão comum (ERC) para os pontos no domínio da imagem,
a partir de $R_N$ e $\beta_0$ obtidos do modelo de velocidades do passo anterior, otimizados a partir do algoritmo
very fast simulated aneeling (VFSA).
As coordenadas $m$ e $h$ e o tempo de trânsito da ERC são interpoladas a partir das aproximações
de tempo de trânsito SRC Padé \cite{neves} que tangencia a superfície de tempo de trânsito ERC.
A partir daí empilhamos sobre a curva de tempo de trânsito ERC no domínio dos dados e ressaltamos as difrações.