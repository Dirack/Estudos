\chapter{METODOLOGIA}
\label{met}

\begin{enumerate}
 \item Partindo da seção empilhada Separar os eventos de reflexão e difração utilizando o filtro de destruição de ondas planas.
 
 \item Partindo da seção empilhada contendo apenas eventos de difração migramos a seção múltiplas vezes utilizando
diferentes velocidades de migração em um processo chamado de velocity continuation, a partir de um algoritmo baseado
na FFT.

\item Montar os FIG (focusing image gathers) calculando para cada ponto i no domínio da imagem o seu Local varimax,
para cada uma das imagens migradas produzidas no passo anterior a partir de uma velocidade de migração distinta. 
Para pontos no domínio da imagem que sejam pontos de foco, o valor do local varimax é máximo, assim o FIG é um painel que mostra os maiores valores de local varimax para determinada velocidade de migração (análise de velocidade de migração).


\end{enumerate}
