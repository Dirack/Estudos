%
% exemplo.tex (LateX)
%
% Versão modificada do exemplo disponível em: 
% https://alvinalexander.com/blog/post/latex/create-your-own-commands-in-latex-using-newcommand 
% 
% Objetivo: Estudo sobre criação de comandos no LateX.
% 
% Versão 1.0
% 
% Site: http://www.dirackslounge.online
% 
% Programador: Rodolfo A. C. Neves (Dirack) 30/06/2019
% 
% Email: rodolfo_profissional@hotmail.com
% 
% Licença: Software de uso livre e código aberto.

\documentclass[a4paper,11pt]{article}
\author{Rodolfo A. C. Neves (Dirack)}
\title{Documento de exemplo}

\usepackage{ifthen}			% Utilização de blocos if else
\usepackage[utf8]{inputenc}		% Determina a codificação utiizada (conversão automática dos acentos)
\usepackage{graphicx}			% Inclusão de gráficos com \include{arquivo}

\begin{document}

% Função que imprime a frase ``Um texto qualquer'' no documento
\newcommand{\printTexto}
{
  Um texto qualquer
}

\printTexto

%-----------------------------------------
% Trocar as tags "enumerate"
%     por comandos próprios
%-----------------------------------------

\newcommand{\be}{\begin{enumerate}}
\newcommand{\ee}{\end{enumerate}}
 
\be
  \item Item 1
  \item Item 2
\ee


%-----------------------------------------------
% Exemplo de renewcommand modificando printTexto
%-----------------------------------------------
\renewcommand{\printTexto}
{
  OUTRO texto qualquer
}
 
\printTexto


% Função que recebe os parâmetros e imprime uma imagem no documento
% #1 - arquivo com a imagem
% #2 - Legenda
% #3 - Label
% #4 - Autor
% #5 - Escala da imagem
\newcommand{\imagex}[5]
 {

   	\begin{figure}[htb]
   	\caption{#2}
  	\begin{center}
   	\includegraphics[scale=#5]{#1}
   	\vspace{-0.3cm}
   	\end{center}
   	\begin{center}
   	 Fonte: #4
   	\end{center}
   	\label{#3}
   	\end{figure}
}

\imagex{images/test.jpg}{Imagem de Exemplo da função ilustrando passagem de parâmetros. 
E definição de macros.}{fig:1}{Do Autor.}{0.5}


\end{document}
