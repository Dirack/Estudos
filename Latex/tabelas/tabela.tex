% 
% 	tabela.tex (LATEX)
% 
% 	Objetivo: Estudo sobre tabelas em Latex.
%
%	Versão 1.0
% 
% 	Programador: Rodolfo A. C. Neves (Dirack) 22/02/2019
% 
%	Licensa: Software de uso livre e gratuito.

\documentclass[a4paper, 12pt]{article}

% Pacotes fundamentais
  \usepackage{multirow}
  \usepackage[top=2cm, bottom=2cm, left=2.5cm, right=2.5cm]{geometry}
  \usepackage[utf8]{inputenc}
  \usepackage{amsmath, amsfonts, amssymb}
  \usepackage{float}
  \usepackage{graphicx}
  \usepackage[portuguese]{babel}

\begin{document}

Estudo sobre Tabelas em LATEX

% Tabela do Latex
%   A estrutura de uma tabela em Latex é cada célula de uma linha é separada por &
%   E as linhas entre si são separadas por \\ 
%   Exemplo:
%   
%     1 & 2 & 3 \\
%     4 & 5 & 6 \\
%     7 & 8 & 9

  \begin{table}[H]
      \caption{Esta é a legenda de uma tabela simples em Latex.}
      \centering % centralizar
      
      \begin{tabular}{ccc} % 3 colunas centralizadas 

      1 & 2 & 3 \\  								
      4 & 5 & 6 \\
      7 & 8 & 9  
      
      \end{tabular}
  \end{table}
  
  
%   Inserir linhas verticais com |c| e horizontais com \hline
    \begin{table}[H]
      \caption{Esta é a legenda de uma tabela simples em Latex com linhas verticais e horizontais.}
      \centering % centralizar
      
      \begin{tabular}{|c|c|c|} % 3 colunas centralizadas separadas por linhas verticais

      \hline
      1 & 2 & 3 \\ \hline  								
      4 & 5 & 6 \\ \hline
      7 & 8 & 9  \\ 
      \hline
      
      \end{tabular}
  \end{table}

% Dá para expandir uma célula com o pacote multirow

  \begin{table}[H]
      \caption{Esta é a legenda da tabela em que a célula um se expande por duas linhas 
      utilizando o pacote multirow.}
      \centering % centralizar
      
      \begin{tabular}{|c|c|c|} % 3 colunas centralizadas separadas por linhas verticais

      \hline % Linha horizontal
      
      % \cline{i-j} insere uma linha horizontal
      % entre as colunas i e j
      % multirow{2}{*}{item 1} expande a célula 1 em duas linhas
      \multirow{2}{*}{Multirow} & Item 2 & Item 3 \\ \cline{2-3} 								
      & 1 & 2\\ \hline
      
      \end{tabular}
  \end{table}
  
    \begin{table}[H]
      \caption{Esta é a legenda da tabela em que a célula um se expande por duas colunas 
      utilizando o pacote multicolumn.}
      \centering % centralizar
      
      \begin{tabular}{|c|c|c|} % 3 colunas centralizadas separadas por linhas verticais

      \hline % Linha horizontal
      
      % multicolumn{2}{|c|}{item 1} expande a célula 1 em duas colunas
      \multicolumn{2}{|c|}{Multicolumn} & 2\\ \hline
      1 & 2 & 3\\
      \hline
      
      \end{tabular}
  \end{table}

  
  %% Exemplo de como controlar o tamanho da coluna da tabela
    \begin{table}[H]
      \caption{Controlar o tamanho da coluna da tabela.}
      \centering % centralizar
      
      \begin{tabular}{|p{2.5cm}|p{2cm}|p{1.5cm}|p{1cm}|p{0.5cm}|} % 5 colunas centralizadas 

      \hline
      1 & 2 & 3 & 4 & 5\\  \hline
      4 & 5 & 6 & 7 & 8\\ \hline
      7 & 8 & 9 & 10 & 11\\  \hline
      
      \end{tabular}
  \end{table}
  
  
\end{document}
