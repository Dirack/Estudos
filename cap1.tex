\chapter{MÉTODO crs}
\label{cap1}


Sendo desenhado para refletores pouco inclinados e pequenas variações laterais de velocidade da subsuperfície, e
para pequenos afastamentos fonte receptor as correções de sobretempo nem sempre são acuradas quando estas condições
são severamente violadas. \cite{tygel}

Na aquisição moderna, os dados sísmicos organizados em famílias de ponto médio comum (PMC) representam apenas uma fração
dos dados adquiridos. Como consequência, as expressões de sobretempo que usam posições arbitrárias dos pares fonte receptor
ao longo de um ponto central fixo (que pode ser inclusive um PMC) são aptas a fornecer melhor uso dos dados disponíveis
e usufruir da grande redundância que é oferecida. \cite{tygel}

O operador do empilhaento SRC é baseado em três atributos de duas frentes de ondas chamadas de eingenwaves
relacionadas ao raio de incidência normal \cite{hubral}.

O empilhamento por superfície de reflexão comum provém uma seção de afastamento nulo simulada a partir de dados sísmicos
de reflexão de multicobertura. Enquanto os métodos de imageamento convencionais requerem um macromodelo de velocidades
acurado para levar a resultados apropriados, o empilhamento por superfície de reflexão comum (SRC) não depende de um
macromodelo de velocidades \cite{jager}.

Para a aquisição 2D, o empilhamento SRC produz uma superfície de empilhamento dependendente de três parâmetros.
A superfície de empilhamento ótima precisa ser determinada para cada ponto da seção de afastamento nulo simulada.
Para uma dada reflexão primária, estes parâmetros são o ângulo de emergência $\beta$ do raio de afastamento nulo, bem como
os raios de curvatura das frentes de onda $R_N$ e $R_{NIP}$. Estes estão associados com duas ondas hipotéticas:
A onda normal (N) e a onda de ponto de incidência normal (NIP) \cite{jager}.

Para cada ponto $X_0$ na seção de afastamento nulo simulada com o empilhamento SRC, temos que determinar a tripla de
parâmetros ótimos ($\beta$, $R_N$ e $R_{NIP}$), a tripla para o qual o operador do empilhamento SRC ajusta melhor
os eventos de reflexão no domínio dos dados \cite{jager}.

A decomposição do campo de onda total sobre um conjunto de traços em partes correspondentes a diferentes ondas
de corpo é um dos problemas fundamentais do processamento de dados sísmicos. 
Em ordem de implementar esse procedimento é necessário
ter uma fórmula local de correção do tempo para uma família de pares fonte receptor distribuídos em volta de volta
de um par central escolhido.\cite{gelpart1}

De um ponto de vista físico, é possível corrigir todos os traços empilhados antes do empilhamento de uma maneira
que todos os eventos alvo estarão em fase. Assim, a operação crucial no empilhamento é a correção de tempo.\cite{gelpart1}
