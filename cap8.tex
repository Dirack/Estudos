\chapter{CONCLUSÃO}
\label{cap8}

As equações do tempo de trânsito utilizando aproximantes de Padé 
surgem como alternativa para aproximar o tempo de trânsito em grandes afastamentos
são mais acuradas do que a aproximação hiperbólica do tempo de
trânsito (sobretempo normal). Estas aproximações dependem apenas de 3 parâmetros,  $v$,  $t_0$ e o parâmetro $A$,
obtido
através da expansão em série de Taylor de quarta ordem do tempo de trânsito ao longo do raio normal.

As aproximações do SRC utilizando aproximantes de Padé, são mais acuradas que a aproximação hiperbólica do SRC. 
As aproximações
não hiperbólicas do SRC: SRC Padé, SRC não hiperbólico e SRC quarta ordem, estendem a região de convergência da aproximação
hiperbólica do SRC para maiores distâncias entre os pontos médios e meio afastamentos, surgindo como alternativa 
para a realização do empilhamento através de superfícies SRC não hiperbólicas que estendem a região de empilhamento,
aumentando a cobertura e razão sinal/ruído.

A aproximação \textit{quadrática} do SRC quarta ordem produziu o melhor ajuste na inversão dos parâmetros do SRC
através da otimização por mínimos quadrados. 
Isso sugere que, apesar do método de inversão utilizado
tornar complicada a utilização das aproximações SRC Padé por causa da dificuldade
em calcular as derivadas parciais
necessárias, utilizando outro método de inversão dos parâmetro do SRC que não necessite dessas
derivadas,
a região de convergência obtida tenderia a aumentar através da introdução das aproximações SRC Padé.


Como sugestão para pesquisas futuras propomos a realização das etapas de empilhamento, tanto para a configuração 
ponto médio comum quanto SRC, sobre as curvas e superfícies produzidas através das aproximações de Padé. 
Enfatizando que as aberturas no domínio do afastamento e dos pontos médios irá influênciar na diferença
de acurácia nas seções empilhadas. Uma vez, que a melhora na acurácia das aproximações não hiperbólicas em
relação as aproximações sobretempo normal e SRC hiperbólico, surge
para grandes afastamentos e distâncias entre os pontos médios.
O parâmetro $A$ pode
ser obtido através de \textit{Semblance}, para encontrar o parâmetro
que melhor ajusta a curva aos dados para grandes afastamentos, a velocidade
utilizada na aproximação será a $v_{RMS}$. No caso SRC, uma vez obtidos os parâmetros $R_N$,
$R_{NIP}$ e $\beta$ bastará substituí-los nas aproximações SRC Padé e realizar 
o empilhamento sobre as superfícies SRC aproximadas.
